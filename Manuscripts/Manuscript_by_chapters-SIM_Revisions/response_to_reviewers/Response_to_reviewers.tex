% Options for packages loaded elsewhere
\PassOptionsToPackage{unicode}{hyperref}
\PassOptionsToPackage{hyphens}{url}
%
\documentclass[
]{article}
\title{\textbf{Generalized additive models to analyze biomedical non-linear longitudinal data in R:}\\
Beyond repeated measures ANOVA and Linear Mixed Models\\
\strut \\
Response to reviewers}
\author{}
\date{\vspace{-2.5em}}

\usepackage{amsmath,amssymb}
\usepackage{lmodern}
\usepackage{iftex}
\ifPDFTeX
  \usepackage[T1]{fontenc}
  \usepackage[utf8]{inputenc}
  \usepackage{textcomp} % provide euro and other symbols
\else % if luatex or xetex
  \usepackage{unicode-math}
  \defaultfontfeatures{Scale=MatchLowercase}
  \defaultfontfeatures[\rmfamily]{Ligatures=TeX,Scale=1}
\fi
% Use upquote if available, for straight quotes in verbatim environments
\IfFileExists{upquote.sty}{\usepackage{upquote}}{}
\IfFileExists{microtype.sty}{% use microtype if available
  \usepackage[]{microtype}
  \UseMicrotypeSet[protrusion]{basicmath} % disable protrusion for tt fonts
}{}
\makeatletter
\@ifundefined{KOMAClassName}{% if non-KOMA class
  \IfFileExists{parskip.sty}{%
    \usepackage{parskip}
  }{% else
    \setlength{\parindent}{0pt}
    \setlength{\parskip}{6pt plus 2pt minus 1pt}}
}{% if KOMA class
  \KOMAoptions{parskip=half}}
\makeatother
\usepackage{xcolor}
\IfFileExists{xurl.sty}{\usepackage{xurl}}{} % add URL line breaks if available
\IfFileExists{bookmark.sty}{\usepackage{bookmark}}{\usepackage{hyperref}}
\hypersetup{
  hidelinks,
  pdfcreator={LaTeX via pandoc}}
\urlstyle{same} % disable monospaced font for URLs
\usepackage[margin=1in]{geometry}
\usepackage{listings}
\newcommand{\passthrough}[1]{#1}
\lstset{defaultdialect=[5.3]Lua}
\lstset{defaultdialect=[x86masm]Assembler}
\usepackage{longtable,booktabs,array}
\usepackage{calc} % for calculating minipage widths
% Correct order of tables after \paragraph or \subparagraph
\usepackage{etoolbox}
\makeatletter
\patchcmd\longtable{\par}{\if@noskipsec\mbox{}\fi\par}{}{}
\makeatother
% Allow footnotes in longtable head/foot
\IfFileExists{footnotehyper.sty}{\usepackage{footnotehyper}}{\usepackage{footnote}}
\makesavenoteenv{longtable}
\usepackage{graphicx}
\makeatletter
\def\maxwidth{\ifdim\Gin@nat@width>\linewidth\linewidth\else\Gin@nat@width\fi}
\def\maxheight{\ifdim\Gin@nat@height>\textheight\textheight\else\Gin@nat@height\fi}
\makeatother
% Scale images if necessary, so that they will not overflow the page
% margins by default, and it is still possible to overwrite the defaults
% using explicit options in \includegraphics[width, height, ...]{}
\setkeys{Gin}{width=\maxwidth,height=\maxheight,keepaspectratio}
% Set default figure placement to htbp
\makeatletter
\def\fps@figure{htbp}
\makeatother
\setlength{\emergencystretch}{3em} % prevent overfull lines
\providecommand{\tightlist}{%
  \setlength{\itemsep}{0pt}\setlength{\parskip}{0pt}}
\setcounter{secnumdepth}{5}
%\usepackage{lineno}
\usepackage[affil-it,blocks]{authblk}
\usepackage{hyperref}
\usepackage{graphicx}
%\usepackage[nomarkers,figuresonly]{endfloat}
%set a box to put the ORCID logo
\newbox{\myorcidaffilbox}
\sbox{\myorcidaffilbox}{\large\includegraphics[height=1.7ex]{latex_docs/orcid}}

%add hyperlink to the box
\newcommand{\orcidaffila}[1]{%
  \href{https://orcid.org/0000-0002-6014-4538}{\usebox{\myorcidaffilbox}}}

\newcommand{\orcidaffilb}[1]{%
  \href{https://orcid.org/0000-0002-6135-8191}{\usebox{\myorcidaffilbox}}}


%to have page numbers with letters in the Appendix:
% %solution from https://tex.stackexchange.com/questions/59572/custom-page-numbering-for-appendix
\newcommand{\appendixpagenumbering}{
  \break
  \pagenumbering{arabic}
  \renewcommand{\thepage}{\thesection-\arabic{page}}
}

%command for the package lineno
%\linenumbers

%authors
\author{Ariel I. Mundo \orcidaffila{}}
\affil{Department of Biomedical Engineering, University of Arkansas, Fayetteville, AR, USA}
\author{John R. Tipton \orcidaffilb{}}
\affil{Department of Mathematical Sciences, University of Arkansas, Fayetteville, AR, USA}
\author{Timothy J. Muldoon*}
\affil{Department of Biomedical Engineering, University of Arkansas, Fayetteville, AR, USA}
\affil{tmuldoon@uark.edu}


%theme colors for the code chunks (originally from latex-solarized on GitHub)
%https://github.com/jez/latex-solarized
\usepackage{xcolor}
\definecolor{sbase03}{HTML}{002B36}
\definecolor{sbase02}{HTML}{073642}
\definecolor{sbase01}{HTML}{586E75}
\definecolor{sbase00}{HTML}{657B83}
\definecolor{sbase0}{HTML}{839496}
\definecolor{sbase1}{HTML}{93A1A1}
\definecolor{sbase2}{HTML}{EEE8D5}
\definecolor{sbase3}{HTML}{FDF6E3}
\definecolor{syellow}{HTML}{B58900}
\definecolor{sorange}{HTML}{CB4B16}
\definecolor{sred}{HTML}{DC322F}
\definecolor{smagenta}{HTML}{D33682}
\definecolor{sviolet}{HTML}{6C71C4}
\definecolor{sblue}{HTML}{268BD2}
\definecolor{scyan}{HTML}{2AA198}
\definecolor{sgreen}{HTML}{859900}
%command to set parameter(s) in package listings
\lstset{
    % How/what to match
    sensitive=true,
    % Border (above and below)
    frame=lines,
    % Extra margin on line (align with paragraph)
    xleftmargin=\parindent,
    % Put extra space under caption
    belowcaptionskip=1\baselineskip,
    % Colors
    backgroundcolor=\color{sbase3},
    basicstyle=\color{sbase00}\ttfamily,
    keywordstyle=\color{scyan},
    commentstyle=\color{sbase1},
    stringstyle=\color{sblue},
    numberstyle=\color{sviolet},
    identifierstyle=\color{sbase00},
    % Break long lines into multiple lines?
    breaklines=true,
    % Show a character for spaces?
    showstringspaces=false,
    tabsize=2
}


%\lstset{
%  breaklines=true,
%  stringstyle=\ttfamily,
%  backgroundcolor=\color{gray}
%}
\usepackage{placeins}
\usepackage{subfig}
\usepackage{breqn}
\usepackage[font={small}]{caption}
\ifLuaTeX
  \usepackage{selnolig}  % disable illegal ligatures
\fi

\begin{document}
\maketitle

\hypertarget{general-comments}{%
\section{General Comments}\label{general-comments}}

\hypertarget{comments-from-reviewers}{%
\section{Comments from reviewers}\label{comments-from-reviewers}}

\hypertarget{missing-data}{%
\subsection{Missing data}\label{missing-data}}

The title for Section 3.4 has been changed from ``Missing observations'' to ``Unbalanced data''. We have restructured this section in order to:

\begin{itemize}
\item
  Avoid confusion by using the term ``missing data''. We now refer to different number of observations as ``unbalanced data'' (L201-204).
\item
  Make clear that LMEMs also can work with missing observations (L209)
\item
  Emphasize that GAMs are not immune to missing data problems (L217-220)
\end{itemize}

\hypertarget{unnecesary-restriction-of-lmms}{%
\subsection{Unnecesary restriction of LMMs}\label{unnecesary-restriction-of-lmms}}

\hypertarget{glmms}{%
\subsection{GLMMs}\label{glmms}}

\hypertarget{glms-gams-and-conditional-distributions}{%
\subsection{GLMs, GAMs, and conditional distributions}\label{glms-gams-and-conditional-distributions}}

\hypertarget{thin-plate-regression-splines}{%
\subsection{Thin plate regression splines}\label{thin-plate-regression-splines}}

The following changes have been made:

\begin{itemize}
\item
  The information on L289-294 pertaining splines has been changed. L296-307 now give a better description of cubic splines (CS), indicating their limitation due to knot placement. The lack of knot placement for thin plate regression splines (TPRS) is indicated as an advantage over CS.
\item
  L297-300 no longer refer to ``knots'' when referring to the construction of the smooth, now indicating that the number of \emph{basis functions} is what is specified when working with TPRS. In the updated manuscript this appears in L310-315.
\end{itemize}

\hypertarget{penalised-splines}{%
\subsection{Penalised splines}\label{penalised-splines}}

\hypertarget{bayesian}{%
\subsection{Bayesian}\label{bayesian}}

\begin{itemize}
\item
  We have clarified the concept ``empirical Bayesian'', which appeared on L318-L319 in the original manuscript. In the revised manuscript, L335-336 now indicate that Stan, JAGS or other probabilistic programming language can be used to estimate GAMs using a full Bayesian approach.
\item
  The sentence ``Moreover, the use of the restricted maximum likelihood (REML) to estimate the smoothing parameters gives an empirical estimate of the smooth model'', which appears in L319 in the original manuscript has been removed from the text. Instead, the concept of REML has been moved to Section 6.2 L390-393, where we state some of the reasons indicated by Wood when choosing restricted maximum likelihood (REML) over the default general cross validation (GCV) method for smooth parameter estimation in \emph{mgcv}.
\end{itemize}

\hypertarget{coverage-of-confidence-intervals}{%
\subsection{Coverage of confidence intervals}\label{coverage-of-confidence-intervals}}

The content in L320-329 regarding confidence intervals (CIs) in the original submission has been changed in the following manner:

\begin{itemize}
\tightlist
\item
  L339-L352 now contain a more detailed and accurate explanation on the differences and interpretation of ``pointwise'' CIs and ``across-the-function'' CIs. We also reference the work of Marra and Wood if the reader desires a more in-depth exploration.
\end{itemize}

\hypertarget{differences-in-smooths}{%
\subsection{Differences in smooths}\label{differences-in-smooths}}

\hypertarget{appendix}{%
\subsection{Appendix}\label{appendix}}

\end{document}
