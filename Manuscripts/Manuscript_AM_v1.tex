\PassOptionsToPackage{unicode=true}{hyperref} % options for packages loaded elsewhere
\PassOptionsToPackage{hyphens}{url}
%
\documentclass[
]{article}
\usepackage{lmodern}
\usepackage{amssymb,amsmath}
\usepackage{ifxetex,ifluatex}
\ifnum 0\ifxetex 1\fi\ifluatex 1\fi=0 % if pdftex
  \usepackage[T1]{fontenc}
  \usepackage[utf8]{inputenc}
  \usepackage{textcomp} % provides euro and other symbols
\else % if luatex or xelatex
  \usepackage{unicode-math}
  \defaultfontfeatures{Scale=MatchLowercase}
  \defaultfontfeatures[\rmfamily]{Ligatures=TeX,Scale=1}
  \setmainfont[]{Times New Roman}
\fi
% use upquote if available, for straight quotes in verbatim environments
\IfFileExists{upquote.sty}{\usepackage{upquote}}{}
\IfFileExists{microtype.sty}{% use microtype if available
  \usepackage[]{microtype}
  \UseMicrotypeSet[protrusion]{basicmath} % disable protrusion for tt fonts
}{}
\makeatletter
\@ifundefined{KOMAClassName}{% if non-KOMA class
  \IfFileExists{parskip.sty}{%
    \usepackage{parskip}
  }{% else
    \setlength{\parindent}{0pt}
    \setlength{\parskip}{6pt plus 2pt minus 1pt}}
}{% if KOMA class
  \KOMAoptions{parskip=half}}
\makeatother
\usepackage{xcolor}
\IfFileExists{xurl.sty}{\usepackage{xurl}}{} % add URL line breaks if available
\IfFileExists{bookmark.sty}{\usepackage{bookmark}}{\usepackage{hyperref}}
\hypersetup{
  pdftitle={Bayesian statistics for repeated measures},
  pdfauthor={Ariel Mundo, John Tipton, Timothy Muldoon},
  pdfborder={0 0 0},
  breaklinks=true}
\urlstyle{same}  % don't use monospace font for urls
\usepackage[margin=1in]{geometry}
\usepackage{graphicx,grffile}
\makeatletter
\def\maxwidth{\ifdim\Gin@nat@width>\linewidth\linewidth\else\Gin@nat@width\fi}
\def\maxheight{\ifdim\Gin@nat@height>\textheight\textheight\else\Gin@nat@height\fi}
\makeatother
% Scale images if necessary, so that they will not overflow the page
% margins by default, and it is still possible to overwrite the defaults
% using explicit options in \includegraphics[width, height, ...]{}
\setkeys{Gin}{width=\maxwidth,height=\maxheight,keepaspectratio}
\setlength{\emergencystretch}{3em}  % prevent overfull lines
\providecommand{\tightlist}{%
  \setlength{\itemsep}{0pt}\setlength{\parskip}{0pt}}
\setcounter{secnumdepth}{-2}
% Redefines (sub)paragraphs to behave more like sections
\ifx\paragraph\undefined\else
  \let\oldparagraph\paragraph
  \renewcommand{\paragraph}[1]{\oldparagraph{#1}\mbox{}}
\fi
\ifx\subparagraph\undefined\else
  \let\oldsubparagraph\subparagraph
  \renewcommand{\subparagraph}[1]{\oldsubparagraph{#1}\mbox{}}
\fi

% set default figure placement to htbp
\makeatletter
\def\fps@figure{htbp}
\makeatother


\title{Bayesian statistics for repeated measures}
\usepackage{etoolbox}
\makeatletter
\providecommand{\subtitle}[1]{% add subtitle to \maketitle
  \apptocmd{\@title}{\par {\large #1 \par}}{}{}
}
\makeatother
\subtitle{Their application and use in biomedical research}
\author{Ariel Mundo, John Tipton, Timothy Muldoon}
\date{}

\begin{document}
\maketitle

\hypertarget{paper-outline}{%
\subsection{Paper outline}\label{paper-outline}}

{I have been thinking that this type of paper probably does not need a
relatively long introduction as the one I did for \(JBO\). Below are
some of my ideas on how to tackle this paper, your comments would be
appreciated. Please keep in mind that I have started to write the
Background/Intro part, and just have ideas on how to go from there}

Background/Introduction Background
==============================================================

The study of the temporal changes in a variable of interest (a
longitudinal analysis) is a question that has been analyzed extensively
in biomedical research. Typically, measurements are taken across
multiple timepoints on the same subject(s) within a group. Examples of
this type of approach include clinical studies on cancer breast and neck
cancer(Sio et al. \protect\hyperlink{ref-sio2016}{2016}; Kamstra et al.
\protect\hyperlink{ref-kamstra2015}{2015}), tumor response(Roblyer et
al. \protect\hyperlink{ref-roblyer2011}{2011}; Tank et al.
\protect\hyperlink{ref-tank2020}{2020}; Pavlov et al.
\protect\hyperlink{ref-pavlov2018}{2018}; Demidov et al.
\protect\hyperlink{ref-demidov2018}{2018}), antibody expression(Ritter
et al. \protect\hyperlink{ref-ritter2001}{2001}; Roth et al.
\protect\hyperlink{ref-roth2017}{2017}), and cell metabolism(Jones et
al. \protect\hyperlink{ref-jones2018}{2018}; Skala et al.
\protect\hyperlink{ref-skala2010}{2010}). Whereas this type of study
presents advantages over a cross-sectional study in the number of
subjects required to achieve a certain statistical power, its ability to
provide a time-correlated view of the phenomenon of interest, and, the
potential to explore multiple relationships between different variables,
the statistical analysis of such data is more challenging and requires
careful consideration.

Biomedical researchers typically employ a \(frequentist\) approach to
analyze longitudinal data. Such type of analysis is a hypothesis test
using the \(analysis of variance over repeated measures\) (repeated
measures ANOVA or rm-ANOVA). Although the usual approach, rm-ANOVA is
subject to multiple conditions to be valid, some of which are not easily
verifiable. Moreover, this approach restricts the inferences it can
extract from a longitudinal study, particularly when the data does not
follow a linear trend (Figure 1. with line and ``wiggly plot''), due to
the inherent nature of the model and its limitations .Examples of such
type of data are found in studies of tumor response to
radio/chemotherapy in multiple models and clinical settings
{[}(Vishwanath et al. \protect\hyperlink{ref-vishwanath2009}{2009}){]}.
Therefore, this study has three goals: a) Present in an amenable and
practical manner the requisities of a \(frequentist\) approach over
longitudinal data and how these limit the analysis b) present a
different area of statistical analysis for longitudinal biomedical data
\(Bayesian Statistics\), which does not use \emph{p-values}, and is less
restrictive thereby allowing a statical analysis that is based on the
data itself. c)Iimplement b) over a set of simulated data that matches
previously reported trends in longitudinal biomedical studies. With an
emphasis on reproducibility by providing the code and dataset used, this
will provide biomedical researchers a clear view of the advantages of
Bayesian statistics for the analysis of longitudinal data.

\begin{itemize}
\item
  Paragraph 2: Challenges presented by longitudinal studies. Missing
  observations, and correlation between measurements. Limitations that
  these items raise for the traditional ANOVA methods of analysis.
\item
  Paragraph 3: Bayesian statistics as an alternative approach.
  Advantages over ANOVA and what inference can be made from it. Argue
  that while it is not commonly used in the biomedical arena, this paper
  aims at showing the implementation in an amenable manner to analyze
  non-linear trends in data.
\end{itemize}

\begin{center}\rule{0.5\linewidth}{0.5pt}\end{center}

\begin{itemize}
\item
  Recap on repeated measures ANOVA and the requisites that it needs to
  work properly -Sphericity -Variance-covariance matrix -Explain the
  \emph{true} meaning of a \emph{p-value} and why it is not what
  researchers commonly think it is
\item
  Gentle introduction to Bayesian statistics -How it works -What
  advantages it has over ANOVA -What a confidence interval means
  probabilistically
\end{itemize}

(Chavalarias et al. \protect\hyperlink{ref-chavalarias2016}{2016})

\hypertarget{references}{%
\section*{References}\label{references}}
\addcontentsline{toc}{section}{References}

\hypertarget{refs}{}
\leavevmode\hypertarget{ref-chavalarias2016}{}%
Chavalarias, David, Joshua David Wallach, Alvin Ho Ting Li, and John PA
Ioannidis. 2016. ``Evolution of Reporting P Values in the Biomedical
Literature, 1990-2015.'' \emph{Jama} 315 (11): 1141--8.

\leavevmode\hypertarget{ref-demidov2018}{}%
Demidov, Valentin, Azusa Maeda, Mitsuro Sugita, Victoria Madge,
Siddharth Sadanand, Costel Flueraru, and I Alex Vitkin. 2018.
``Preclinical Longitudinal Imaging of Tumor Microvascular
Radiobiological Response with Functional Optical Coherence Tomography.''
\emph{Scientific Reports} 8 (1): 1--12.

\leavevmode\hypertarget{ref-jones2018}{}%
Jones, Jake D, Hallie E Ramser, Alan E Woessner, and Kyle P Quinn. 2018.
``In Vivo Multiphoton Microscopy Detects Longitudinal Metabolic Changes
Associated with Delayed Skin Wound Healing.'' \emph{Communications
Biology} 1 (1): 1--8.

\leavevmode\hypertarget{ref-kamstra2015}{}%
Kamstra, JI, PU Dijkstra, M Van Leeuwen, JLN Roodenburg, and JA
Langendijk. 2015. ``Mouth Opening in Patients Irradiated for Head and
Neck Cancer: A Prospective Repeated Measures Study.'' \emph{Oral
Oncology} 51 (5): 548--55.

\leavevmode\hypertarget{ref-pavlov2018}{}%
Pavlov, Mikhail V, Tatiana I Kalganova, Yekaterina S Lyubimtseva,
Vladimir I Plekhanov, German Yurievich Golubyatnikov, Olga Y Ilyinskaya,
Anna G Orlova, et al. 2018. ``Multimodal Approach in Assessment of the
Response of Breast Cancer to Neoadjuvant Chemotherapy.'' \emph{Journal
of Biomedical Optics} 23 (9): 091410.

\leavevmode\hypertarget{ref-ritter2001}{}%
Ritter, Gerd, Leonard S Cohen, Clarence Williams, Elizabeth C Richards,
Lloyd J Old, and Sydney Welt. 2001. ``Serological Analysis of Human
Anti-Human Antibody Responses in Colon Cancer Patients Treated with
Repeated Doses of Humanized Monoclonal Antibody A33.'' \emph{Cancer
Research} 61 (18): 6851--9.

\leavevmode\hypertarget{ref-roblyer2011}{}%
Roblyer, Darren, Shigeto Ueda, Albert Cerussi, Wendy Tanamai, Amanda
Durkin, Rita Mehta, David Hsiang, et al. 2011. ``Optical Imaging of
Breast Cancer Oxyhemoglobin Flare Correlates with Neoadjuvant
Chemotherapy Response One Day After Starting Treatment.''
\emph{Proceedings of the National Academy of Sciences} 108 (35):
14626--31.

\leavevmode\hypertarget{ref-roth2017}{}%
Roth, Eli M, Anne C Goldberg, Alberico L Catapano, Albert Torri, George
D Yancopoulos, Neil Stahl, Aurélie Brunet, Guillaume Lecorps, and Helen
M Colhoun. 2017. ``Antidrug Antibodies in Patients Treated with
Alirocumab.''

\leavevmode\hypertarget{ref-sio2016}{}%
Sio, Terence T, Pamela J Atherton, Brandon J Birckhead, David J
Schwartz, Jeff A Sloan, Drew K Seisler, James A Martenson, et al. 2016.
``Repeated Measures Analyses of Dermatitis Symptom Evolution in Breast
Cancer Patients Receiving Radiotherapy in a Phase 3 Randomized Trial of
Mometasone Furoate Vs Placebo (N06c4 {[}Alliance{]}).'' \emph{Supportive
Care in Cancer} 24 (9): 3847--55.

\leavevmode\hypertarget{ref-skala2010}{}%
Skala, Melissa C, Andrew Nicholas Fontanella, Lan Lan, Joseph A Izatt,
and Mark W Dewhirst. 2010. ``Longitudinal Optical Imaging of Tumor
Metabolism and Hemodynamics.'' \emph{Journal of Biomedical Optics} 15
(1): 011112.

\leavevmode\hypertarget{ref-tank2020}{}%
Tank, Anup, Hannah M Peterson, Vivian Pera, Syeda Tabassum, Anais
Leproux, Thomas O'Sullivan, Eric Jones, et al. 2020. ``Diffuse Optical
Spectroscopic Imaging Reveals Distinct Early Breast Tumor Hemodynamic
Responses to Metronomic and Maximum Tolerated Dose Regimens.''
\emph{Breast Cancer Research} 22 (1): 1--10.

\leavevmode\hypertarget{ref-vishwanath2009}{}%
Vishwanath, Karthik, Hong Yuan, William T Barry, Mark W Dewhirst, and
Nimmi Ramanujam. 2009. ``Using Optical Spectroscopy to Longitudinally
Monitor Physiological Changes Within Solid Tumors.'' \emph{Neoplasia} 11
(9): 889--900.

\end{document}
