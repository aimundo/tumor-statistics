% Options for packages loaded elsewhere
\PassOptionsToPackage{unicode}{hyperref}
\PassOptionsToPackage{hyphens}{url}
%
\documentclass[
]{article}
\usepackage{lmodern}
\usepackage{amssymb,amsmath}
\usepackage{ifxetex,ifluatex}
\ifnum 0\ifxetex 1\fi\ifluatex 1\fi=0 % if pdftex
  \usepackage[T1]{fontenc}
  \usepackage[utf8]{inputenc}
  \usepackage{textcomp} % provide euro and other symbols
\else % if luatex or xetex
  \usepackage{unicode-math}
  \defaultfontfeatures{Scale=MatchLowercase}
  \defaultfontfeatures[\rmfamily]{Ligatures=TeX,Scale=1}
  \setmainfont[]{Times New Roman}
\fi
% Use upquote if available, for straight quotes in verbatim environments
\IfFileExists{upquote.sty}{\usepackage{upquote}}{}
\IfFileExists{microtype.sty}{% use microtype if available
  \usepackage[]{microtype}
  \UseMicrotypeSet[protrusion]{basicmath} % disable protrusion for tt fonts
}{}
\makeatletter
\@ifundefined{KOMAClassName}{% if non-KOMA class
  \IfFileExists{parskip.sty}{%
    \usepackage{parskip}
  }{% else
    \setlength{\parindent}{0pt}
    \setlength{\parskip}{6pt plus 2pt minus 1pt}}
}{% if KOMA class
  \KOMAoptions{parskip=half}}
\makeatother
\usepackage{xcolor}
\IfFileExists{xurl.sty}{\usepackage{xurl}}{} % add URL line breaks if available
\IfFileExists{bookmark.sty}{\usepackage{bookmark}}{\usepackage{hyperref}}
\hypersetup{
  pdftitle={The statistical analysis of non-linear longitudinal data in biomedical research},
  pdfauthor={Ariel Mundo; Timothy J. Muldoon; John R. Tipton},
  hidelinks,
  pdfcreator={LaTeX via pandoc}}
\urlstyle{same} % disable monospaced font for URLs
\usepackage[margin=1in]{geometry}
\usepackage{graphicx,grffile}
\makeatletter
\def\maxwidth{\ifdim\Gin@nat@width>\linewidth\linewidth\else\Gin@nat@width\fi}
\def\maxheight{\ifdim\Gin@nat@height>\textheight\textheight\else\Gin@nat@height\fi}
\makeatother
% Scale images if necessary, so that they will not overflow the page
% margins by default, and it is still possible to overwrite the defaults
% using explicit options in \includegraphics[width, height, ...]{}
\setkeys{Gin}{width=\maxwidth,height=\maxheight,keepaspectratio}
% Set default figure placement to htbp
\makeatletter
\def\fps@figure{htbp}
\makeatother
\setlength{\emergencystretch}{3em} % prevent overfull lines
\providecommand{\tightlist}{%
  \setlength{\itemsep}{0pt}\setlength{\parskip}{0pt}}
\setcounter{secnumdepth}{-\maxdimen} % remove section numbering

\title{\textbf{The statistical analysis of non-linear longitudinal data in
biomedical research}}
\usepackage{etoolbox}
\makeatletter
\providecommand{\subtitle}[1]{% add subtitle to \maketitle
  \apptocmd{\@title}{\par {\large #1 \par}}{}{}
}
\makeatother
\subtitle{\emph{Beyond repeated measures ANOVA and Linear Mixed Models}}
\author{Ariel Mundo\footnote{Department of Biomedical Engineering, University of
  Arkansas, Fayetteville} \and Timothy J. Muldoon\footnote{Department of Biomedical Engineering,
  University of Arkansas, Fayetteville} \and John R. Tipton\footnote{Department of Mathematical Sciences, University
  of Arkansas, Fayetteville}}
\date{}

\begin{document}
\maketitle

\hypertarget{paper-outline}{%
\subsection{Paper outline}\label{paper-outline}}

{\textcolor{red}{The paper Introduction has been updated, proposed sections appear at the end of the document as well as an initial graph.}}

\hypertarget{background}{%
\subsection{Background}\label{background}}

A longitudinal study is defined as that which is designed to repeatedly
measure a variable of interest in a group (or groups) of subjects. In
biomedical research, this type of study is employed when the intention
is to observe the evolution of the effect of a certain treatment across
time, rather than analyzing it at a single time point (a cross-sectional
study). Clinical studies of breast and neck cancer have used a
longitudinal approach {[}\protect\hyperlink{ref-sio2016}{1}{]},
{[}\protect\hyperlink{ref-kamstra2015}{2}{]}. In the first case, weekly
measurements of skin toxicities in patients with radiation-induced
dermatitis were taken for up to 8 weeks; whereas in the latter mouth
opening was assessed at 6,12, 18, 24 and 36 months after radiotherapy
(RT). Longitudinal studies have been also used to analyze tumor response
{[}\protect\hyperlink{ref-roblyer2011}{3}{]}--{[}\protect\hyperlink{ref-demidov2018}{6}{]},
antibody expression {[}\protect\hyperlink{ref-ritter2001}{7}{]},
{[}\protect\hyperlink{ref-roth2017}{8}{]}, and cell metabolism
{[}\protect\hyperlink{ref-jones2018}{9}{]},
{[}\protect\hyperlink{ref-skala2010}{10}{]}.

Traditionally, a ``frequentist'' approach is used in biomedical research
to derive inferences from a longitudinal study. Such statistical view
derives its name from the fact that it regards probability as a limiting
frequency{[}\protect\hyperlink{ref-wagenmakers2008}{11}{]} and its
application is based on a null hypothesis test using the \emph{analysis
of variance over repeated measures} (repeated measures ANOVA or
rm-ANOVA). This methodology makes two key assumptions regarding
longitudinal data: constant correlation exists across same-subject
measurements, and observations from each subject are obtained at all
time points through the study (a condition also known as complete
observations) {[}\protect\hyperlink{ref-gueorguieva2004}{12}{]},
{[}\protect\hyperlink{ref-schober2018}{13}{]}.

However, constant correlation is frequently unjustified as its value
tends to diminish between measures when the time interval between them
increases {[}\protect\hyperlink{ref-ugrinowitsch2004}{14}{]}. In the
case of the number of observations during the study, different
situations can arise that prevent the collection of complete
measurements for all subjects: in a clinical trial voluntary withdrawal
from one or multiple patients can occur, whereas attrition in animals
due to injury or weight loss can occur in preclinical experiments, and
it is even possible that unexpected complications with equipment or
supplies arise, preventing the researcher from collecting measurements
at a certain time point and therefore violating the \emph{complete
observations} assumption of rm-ANOVA.

When incomplete observations occur, an rm-ANOVA requires the exclusion
of all subjects with missing observations from the analysis. This can
result in increased costs if the desired statistical power is not met
with the remaining observations, as it would be necessary to enroll more
subjects. At the same time, if the excluded observations contain
insightful information that is not used, their elimination from the
analysis may limit the demonstration of significant differences between
groups. Additionally, rm-ANOVA uses a \emph{post hoc} analysis to assess
differences between the measured response in different groups. A
\emph{post hoc} analysis is based on multiple repeated comparisons to
estimate a \emph{p-value}, a metric that is widely used as a measure of
significance. Because the \emph{p-value} is highly variable, multiple
comparisons can inflate the false positivity rate
{[}\protect\hyperlink{ref-liu2010}{15}{]},
{[}\protect\hyperlink{ref-halsey2015}{16}{]}, consequently biasing the
conclusions of the study.

During the last decade, the biomedical community has started to
recognize the limitations of rm-ANOVA in the analysis of longitudinal
information. This is exemplified by the use of linear mixed effects
models (LMEMs) in certain groups to analyze longitudinal data
{[}\protect\hyperlink{ref-skala2010}{10}{]},
{[}\protect\hyperlink{ref-vishwanath2009}{17}{]}. Briefly, these models
incorporate \emph{fixed effects}, which correspond to the levels of
experimental factors in the study (e.g.~the different drug regimens in a
clinical trial), and \emph{random effects}, which account for random
variation within the population. These models are more flexible than the
traditional rm-ANOVA as they can accommodate missing observations for
multiple subjects and allow different modeling strategies for the
variability within each measure in every subject
{[}\protect\hyperlink{ref-pinheiro2006}{18}{]}. On the other hand, they
impose restrictions in the distribution of the errors of the random
effects {[}\protect\hyperlink{ref-gueorguieva2004}{12}{]}.

One final assumption that is not initially evident for both rm-ANOVA and
LMEMs models pertains the relationship between the observed effect and
the progression of the timeline of the study, which is expected to be
linear {[}\protect\hyperlink{ref-pinheiro2006}{18}{]}. This common
assumption in both rm-ANOVA ane LMEMs consequently restricts the
inferences from the model when the data does not follow a linear trend.
In biomedical research, a special case of this behavior in longitudinal
data arises in measurements of tumor response in preclinical and
clinical settings {[}\protect\hyperlink{ref-roblyer2011}{3}{]},
{[}\protect\hyperlink{ref-skala2010}{10}{]},
{[}\protect\hyperlink{ref-vishwanath2009}{17}{]}. These studies have
shown that the collected signal does not follow a linear trend over
time, and presents extreme variability at different time points, making
the fit of and the estimations of an LMEM or rm-ANOVA model inconsistent
with the observed variation. In other words, the in both cases the model
tries to explain highly-variable data using a linear trend, consequently
biasing the estimates. Additionally, although it is possible that a
\emph{post hoc} analysis is able to find ``significant''
\emph{p-values}( \emph{p}\textless0.05) by using multiple comparisons
between the model terms, the use of this estimator is limited in such
scenario because it is intended to work in models that have a reasonable
agreement with the data.

As the ``frequentist'' rm-ANOVA and the more advanced LMEM approach are
both limited in the analysis of non-linear longitudinal information,
there is a need for biomedical researchers to explore the use of
additional statistical tools that allow the information (and not an
assumed trend) to determine the fit of the model, while enabling
inferences that are both adequate and consistent from a statistical
perspective. In this regard, Generalized Additive Models (GAMs) present
an alternative approach to analyze longitudinal data. Although not
commonly used in the biomedical community, these non-parametric models
have been used to analyze temporal variations in geochemical and
palaeoecological data
{[}\protect\hyperlink{ref-rose2012}{19}{]}--{[}\protect\hyperlink{ref-simpson2018}{21}{]},
health-environment interactions
{[}\protect\hyperlink{ref-yang2012}{22}{]} and political science
{[}\protect\hyperlink{ref-beck1998}{23}{]} . There are several
advantages of GAMs over LMEMs and rm-ANOVA models: They have an
extensively developed theoretical background, enable the data to dictate
the trend of the model, can accommodate missing observations and do not
require constant correlation between repeated measurements
{[}\protect\hyperlink{ref-wood2017}{24}{]}.Therefore, GAMs can provide a
more suitable statistical method to analyze non-linear biomedical
longitudinal data.

The current advances in programming languages designed for statistical
analysis (specifically \(\textsf{R}\)), have eased the computational
implementation of more complex models beyond LMEMs. In particular,
\(\textsf{R}\) has an extensive collection of documentation and
functions to fit GAMs in the package \emph{mgcv}
{[}\protect\hyperlink{ref-wood2017}{24}{]}--{[}\protect\hyperlink{ref-wood2016}{26}{]}
that not only speed up the initial stages of the analysis but enable the
use of advanced modeling structures (e.g.~hierarchical models,
confidence interval comparisons) without requiring advanced programming
skills from the user. This programming language is also able to simulate
data, an emerging strategy used to test statistical models
{[}\protect\hyperlink{ref-haverkamp2017}{27}{]}. This allows to create
and explore different alternatives for analysis without collecting
information in the field, and reduces the time window between experiment
design and its implementation.

Therefore, the purpose of this study is to provide biomedical
researchers with a clear understanding of the theory and the practical
implementation of GAMs to analyze longitudinal data using by focusing on
four areas. First, the limitations of rm-ANOVA and LMEMs to analyze
longitudinal data are contrasted in the cases of missing observations,
assumption of linearity and correlation structures. Secondly, the key
theoretical elements of GAMS are presented without using extensive
derivations or complex mathematical notation that can obscure their
understanding. Thirdly, simulated data that follows the trend of
previously reported values
{[}\protect\hyperlink{ref-vishwanath2009}{17}{]} is used to illustrate
the type of non-linear longitudinal data that occurs in biomedical
research and the implementation of GAMs in such scenario. Finally,
reproducibility is emphasized by providing the code to generate the data
and the implementation of GAMs in \(\textsf{R}\), in conjunction with a
step-by-step guide to fit models of increasing complexity.\\
In summary, the exploration of modern statistical techniques to analyze
longitudinal data may allow biomedical researchers to consider the use
of GAMs instead of rm-ANOVA or LMEMs when the data does not follow a
linear trend, and will also help to improve the standards for
reproducibility in biomedical research.

\hypertarget{section-1}{%
\subsection{Section 1:}\label{section-1}}

\hypertarget{challenges-presented-by-longitudinal-studies}{%
\subsubsection{Challenges presented by longitudinal
studies}\label{challenges-presented-by-longitudinal-studies}}

\hypertarget{the-frequentist-case-for-longitudinal-data}{%
\subsubsection{1 The ``frequentist'' case for longitudinal
data}\label{the-frequentist-case-for-longitudinal-data}}

The \emph{repeated measures analysis of variance} (rm-ANOVA) is the
standard statistical analysis for longitudinal data in biomedical
research, but certain assumptions are necessary to make the model valid.
From a practical view, they can be divided in three areas: linear
relationship between covariates and response, constant correlation
between measurements, and complete observations for all subjects. Each
one of these assumptions is discussed below.

\hypertarget{linear-relationship}{%
\paragraph{1.1 Linear relationship}\label{linear-relationship}}

In a biomedical longitudinal study, two or more groups of subjects
(patients, mice, samples) are subject to a different treatments (e.g.~a
``treatment'' group receives a novel drug vs.~a ``control'' group that
receives a placebo), and measurements from each subject within each
group are collected at specific time points. The collected response has
both \emph{fixed} and \emph{random} components. The \emph{fixed}
component can be understood as a constant value in the response which
the researcher is interested in measuring, i.e, the effect of the novel
drug in the ``treatment'' group.The \emph{random} component can be
defined as ``noise'' caused by some inherent variability within the
study. For example, if the blood concentration of the drug is measured
in certain subjects in the early hours of the morning while others are
measured in the afternoon, it is possible that the difference in the
collection time of the measurement introduces some ``noise'' in the
signal. As their name suggests, this ``random'' variability needs to be
modeled as a variable rather than as a constant value.

Mathematically speaking, a rm-ANOVA model can be written as:

\begin{equation}
  y_{i} = \mu+ \beta_1* time_{t} + \beta_2* treatment_{j} + \beta_3* time_{t} * treatment_{j}+\pi_{ij} +\epsilon_{tij}\\ (\#eq:linear-model)
\end{equation}

In this model \(y_i\) is the response by subject \(i\), which can be
decomposed in a mean value \(\mu\), \emph{fixed effects} of time
(\(time_t\)), treatment (\(time_j\)) and their interaction; a
\emph{random effect} \(\pi_{ij}\) of each subject within each group, and
errors per subject per time per group \(\epsilon_{tij}\). Suppose two
treatments groups are used in the study. Models per group can be
generated if 0 is used for the first treatment group (Group A) and 1 for
the second treatment group (Group B). The linear models then become:

\begin{equation}
  y = \begin{cases}
  \mu + \beta_1*time_{t}+\epsilon_{ti}   & \mbox{if Group A}\\
  \mu + \beta_1 * time_{t} + \beta_2 * treatment_{j} +\beta_3* time_{t} * treatment_{j}        +\pi_{i}+ \epsilon_{tij}  & \mbox{if Group B}\\
  \end{cases}
  (\#eq:ANOVA-by-group)
\end{equation}

Furthermore, @ref(eq:ANOVA-by-group) can be re-written under the
generalized linear model (GLM) framework
{[}\protect\hyperlink{ref-nelder1972}{28}{]}. The main difference in
this case is that a GLM model uses a \emph{linking function} to connect
the observed value to the linear model.

Thus,

\begin{equation}
  y =\mu +\boldsymbol{\beta}_1* f(time) + \beta_2* treatment+\boldsymbol{\beta}_3* f(time)*treatment \\

  y = \begin{cases}
  \mu +\boldsymbol{\beta}_1* f(time)  & \mbox{if Group A} \\
  \mu +\boldsymbol{\beta}_1* f(time)  + \beta_2* treatment + \boldsymbol{\beta}_3* f(time) *     treatment   & \mbox{if Group B}\\
  \end{cases}
  \\
  (\#eq:GLM
\end{equation}

The linear model in @ref(eq:linear-model) is a GLM where
\(f(time) = time\) with the identity function \(f(x) = x\).

Regardless of its presentation, @ref(eq:linear-model),
@ref(eq:ANOVA-by-group) and @ref(eq:GLM) show that the model expects a
linear relationship between the covariates and the response. When the
data does not follow a linear trend, the fit that the model produces
does not accurately represent the changes in the data. To exemplify this
consider simulated longitudinal data where the a normally distributed
response of two groups of 10 subjects each is obtained, and an rm-ANOVA
model such as @ref(eq:linear-model) is fitted to the data (code for the
simulated data is available in the Appendix).

\begin{verbatim}
## 
## Attaching package: 'nlme'
\end{verbatim}

\begin{verbatim}
## The following object is masked from 'package:dplyr':
## 
##     collapse
\end{verbatim}

\includegraphics[width=1\linewidth]{Manuscript_AM_v1_files/figure-latex/unnamed-chunk-3-1}

\hypertarget{covariance-in-rm-anova-and-lmems}{%
\paragraph{1.2 Covariance in rm-ANOVA and
LMEMs}\label{covariance-in-rm-anova-and-lmems}}

In a longitudinal study there is an expected \emph{variance} between
repeated measurements on the same subject, and because repeated measures
occur in the subjects within each group, there is a \emph{covariance}
between measurements at each time point within each group. The
\emph{covariance matrix} (also known as the variance-covariance matrix)
is a matrix that captures the variation between and within subjects in a
longitudinal study{[}\protect\hyperlink{ref-wolfinger1996}{29}{]} (For
an in-depth analysis of the covariance matrix see
{[}\protect\hyperlink{ref-west2014}{30}{]},
{[}\protect\hyperlink{ref-weiss2005}{31}{]}).

In the case of an rm-ANOVA analysis, it is assumed that the covariance
matrix has a specific construction known as \emph{compound symmetry}
(also known as ``sphericity'' or ``circularity''). Under this
assumption, the between-subject variance and within-subject correlation
are constant across time
{[}\protect\hyperlink{ref-weiss2005}{31}{]}--{[}\protect\hyperlink{ref-huynh1976}{33}{]}.
However, it has been shown that this condition is frequently unjustified
because the correlation between measurements tends to change over time
{[}\protect\hyperlink{ref-maxwell2017}{34}{]}; and it is higher between
consecutive measurements
{[}\protect\hyperlink{ref-gueorguieva2004}{12}{]},
{[}\protect\hyperlink{ref-ugrinowitsch2004}{14}{]}. Although corrections
can be made (such as Huyhn-Feldt or Greenhouse-Geisser) the
effectiveness of each correction is limited because it depends on the
size of the sample,the number of repeated
measurements{[}\protect\hyperlink{ref-haverkamp2017}{27}{]}, and they
are not robust if the group sizes are unbalanced
{[}\protect\hyperlink{ref-keselman2001}{35}{]}. In other words, if the
data does not present constant correlation between repeated
measurements, the assumptions required for an rm-ANOVA model are not met
and the use of corrections may still not provide a reasonable adjustment
that makes the model valid.

In the case of LMEMs, one key advantage over rm-ANOVA is that they allow
different structures for the variance-covariance matrix including
exponential, autoregressive of order 1, rational quadratic and others
{[}\protect\hyperlink{ref-pinheiro2006}{18}{]}. Nevertheless, the
analysis required to determine an appropriate variance-covariance
structure for the data can be a long process by itself. Overall, the
spherical assumption for rm-ANOVA may not capture the natural variations
of the correlation in the data, and can bias the inferences from the
analysis.

\hypertarget{missing-observations}{%
\paragraph{1.3 Missing observations}\label{missing-observations}}

Missing observations are an issue that arises frequently in longitudinal
studies. In biomedical research, this situation can be caused by reasons
beyond the control of the investigator {[}molenberghs2004{]}.Dropout
from patients, or attrition or injury in animals are among the reasons
for missing observations. Statistically, missing information can be
classified as \emph{missing at random} (MAR), \emph{missing completely
at random} (MCAR), and \emph{missing not at random} (MNAR)
{[}\protect\hyperlink{ref-weiss2005}{31}{]}. In a MAR scenario, the
pattern of the missing information is related to some variable in the
data, but it is not related to the variable of interest
{[}\protect\hyperlink{ref-scheffer2002}{36}{]}. If the data are MCAR,
this means that the missigness is completely unrelated to the collected
information {[}\protect\hyperlink{ref-potthoff2006}{37}{]}, and in the
case of MNAR the missing values are dependent on their value. An
rm-ANOVA model assumes complete observations for all subjects, and
therefore subjects with one or more missing observations are excluded
from the analysis. This is inconvenient because the remaining subjects
might not accurately represent the population, and statistical power is
affected by this reduction in sample size
{[}\protect\hyperlink{ref-ma2012}{38}{]}.

In the case of LMEMs, inferences from the model are valid when missing
observations in the data exist that are MAR
{[}\protect\hyperlink{ref-west2014}{30}{]}. The pattern of missing
observations can be considered MAR if the missing observations are not
related any of the other variables measured in the study
{[}\protect\hyperlink{ref-maxwell2017}{34}{]}. For example, if attrition
occurs in all mice that had lower weights at the beginning of a
chemotherapy response study, the missing data can be considered MAR
because the missigness is unrelated to other variables of interest.

\emph{What limitations LMEMs have regarding missing observations?}

This section has presented the assumptions of rm-ANOVA to analyze
longitudinal information and its differences when compared to LMEMs
regarding to missing data and the modeling of the covariance matrix. Of
notice, LMEMs offer a more robust and flexible approach than rm-ANOVA
and if the data follows a linear trend, they provide an excellent choice
to derive inferences from a repeated measures study. However, when the
data presents high variability, LMEMs fail to capture the non-linear
trend of the data. To analyze such type of data, we present generalized
additive models (GAMs) as an alternative in the following section.

\hypertarget{gams-as-a-special-case-of-generalized-linear-models}{%
\paragraph{2 GAMs as a special case of Generalized Linear
Models}\label{gams-as-a-special-case-of-generalized-linear-models}}

A GAM is a special case of the Generalized Linear Model (GLM), a
framework that allows for response distributions that do not follow a
normal distribution {[}\protect\hyperlink{ref-wood2017}{24}{]},
{[}\protect\hyperlink{ref-hastie1987}{39}{]}. Following the notation by
Simpson {[}\protect\hyperlink{ref-simpson2018}{21}{]} A GAM model can be
represented as:

\begin{equation}
  y_{t}=\beta_0+f(x_t)+\epsilon_t  
  (\#eq:GAM)
\end{equation}

Where \(y_t\) is the response at time \(t\), \(\beta_0\) is the expected
value at time 0, the change of \(y\) over time is represented by the
function \(f(x_t)\) and \(\epsilon_t\) represents the residuals.

In contrast to rm-ANOVA or LMEMs, GAMs use \emph{smooth functions} to
model the relationship between the covariates and the response. This
approach is more advantageous than that of a LMEMs or rm-ANOVA as it
does not restrict the model to a linear relationship. One possible
function for \(f(x_t)\) is a polynomial, but a major limitation is that
they create a ``global'' fit as they assume that the same relationship
exists everywhere, which can cause problems with the fit
{[}\protect\hyperlink{ref-beck1998}{23}{]}.

To overcome this limitation, the smooth functions in GAMs are
represented using \emph{basis functions} over evenly spaced ranges of
the covariates known as \emph{knots}. The \emph{basis function} used is
a cubic spline, which is a smooth curve constructed from cubic
polynomials joined
together{[}\protect\hyperlink{ref-simpson2018}{21}{]},
{[}\protect\hyperlink{ref-wood2017}{24}{]}. Cubic splines have a long
history in their use to solve non-parametric statistical problems and
are the default choice to fit GAMs as they are the simplest option to
obtain visual smoothness {[}\protect\hyperlink{ref-wegman1983}{40}{]}.

\begin{center}\rule{0.5\linewidth}{0.5pt}\end{center}

\hypertarget{references}{%
\section*{References}\label{references}}
\addcontentsline{toc}{section}{References}

\hypertarget{refs}{}
\leavevmode\hypertarget{ref-sio2016}{}%
{[}1{]} T. T. Sio \emph{et al.}, ``Repeated measures analyses of
dermatitis symptom evolution in breast cancer patients receiving
radiotherapy in a phase 3 randomized trial of mometasone furoate vs
placebo (n06c4 {[}alliance{]}),'' \emph{Supportive Care in Cancer}, vol.
24, no. 9, pp. 3847--3855, 2016.

\leavevmode\hypertarget{ref-kamstra2015}{}%
{[}2{]} J. Kamstra, P. Dijkstra, M. Van Leeuwen, J. Roodenburg, and J.
Langendijk, ``Mouth opening in patients irradiated for head and neck
cancer: A prospective repeated measures study,'' \emph{Oral Oncology},
vol. 51, no. 5, pp. 548--555, 2015.

\leavevmode\hypertarget{ref-roblyer2011}{}%
{[}3{]} D. Roblyer \emph{et al.}, ``Optical imaging of breast cancer
oxyhemoglobin flare correlates with neoadjuvant chemotherapy response
one day after starting treatment,'' \emph{Proceedings of the National
Academy of Sciences}, vol. 108, no. 35, pp. 14626--14631, 2011.

\leavevmode\hypertarget{ref-tank2020}{}%
{[}4{]} A. Tank \emph{et al.}, ``Diffuse optical spectroscopic imaging
reveals distinct early breast tumor hemodynamic responses to metronomic
and maximum tolerated dose regimens,'' \emph{Breast cancer research},
vol. 22, no. 1, pp. 1--10, 2020.

\leavevmode\hypertarget{ref-pavlov2018}{}%
{[}5{]} M. V. Pavlov \emph{et al.}, ``Multimodal approach in assessment
of the response of breast cancer to neoadjuvant chemotherapy,''
\emph{Journal of biomedical optics}, vol. 23, no. 9, p. 091410, 2018.

\leavevmode\hypertarget{ref-demidov2018}{}%
{[}6{]} V. Demidov \emph{et al.}, ``Preclinical longitudinal imaging of
tumor microvascular radiobiological response with functional optical
coherence tomography,'' \emph{Scientific reports}, vol. 8, no. 1, pp.
1--12, 2018.

\leavevmode\hypertarget{ref-ritter2001}{}%
{[}7{]} G. Ritter, L. S. Cohen, C. Williams, E. C. Richards, L. J. Old,
and S. Welt, ``Serological analysis of human anti-human antibody
responses in colon cancer patients treated with repeated doses of
humanized monoclonal antibody a33,'' \emph{Cancer Research}, vol. 61,
no. 18, pp. 6851--6859, 2001.

\leavevmode\hypertarget{ref-roth2017}{}%
{[}8{]} E. M. Roth \emph{et al.}, ``Antidrug antibodies in patients
treated with alirocumab,'' 2017.

\leavevmode\hypertarget{ref-jones2018}{}%
{[}9{]} J. D. Jones, H. E. Ramser, A. E. Woessner, and K. P. Quinn, ``In
vivo multiphoton microscopy detects longitudinal metabolic changes
associated with delayed skin wound healing,'' \emph{Communications
biology}, vol. 1, no. 1, pp. 1--8, 2018.

\leavevmode\hypertarget{ref-skala2010}{}%
{[}10{]} M. C. Skala, A. N. Fontanella, L. Lan, J. A. Izatt, and M. W.
Dewhirst, ``Longitudinal optical imaging of tumor metabolism and
hemodynamics,'' \emph{Journal of biomedical optics}, vol. 15, no. 1, p.
011112, 2010.

\leavevmode\hypertarget{ref-wagenmakers2008}{}%
{[}11{]} E.-J. Wagenmakers, M. Lee, T. Lodewyckx, and G. J. Iverson,
``Bayesian versus frequentist inference,'' in \emph{Bayesian evaluation
of informative hypotheses}, Springer, 2008, pp. 181--207.

\leavevmode\hypertarget{ref-gueorguieva2004}{}%
{[}12{]} R. Gueorguieva and J. H. Krystal, ``Move over anova: Progress
in analyzing repeated-measures data andits reflection in papers
published in the archives of general psychiatry,'' \emph{Archives of
general psychiatry}, vol. 61, no. 3, pp. 310--317, 2004.

\leavevmode\hypertarget{ref-schober2018}{}%
{[}13{]} P. Schober and T. R. Vetter, ``Repeated measures designs and
analysis of longitudinal data: If at first you do not succeed---try, try
again,'' \emph{Anesthesia and analgesia}, vol. 127, no. 2, p. 569, 2018.

\leavevmode\hypertarget{ref-ugrinowitsch2004}{}%
{[}14{]} C. Ugrinowitsch, G. W. Fellingham, and M. D. Ricard,
``Limitations of ordinary least squares models in analyzing repeated
measures data,'' \emph{Medicine and science in sports and exercise},
vol. 36, pp. 2144--2148, 2004.

\leavevmode\hypertarget{ref-liu2010}{}%
{[}15{]} C. Liu, T. P. Cripe, and M.-O. Kim, ``Statistical issues in
longitudinal data analysis for treatment efficacy studies in the
biomedical sciences,'' \emph{Molecular Therapy}, vol. 18, no. 9, pp.
1724--1730, 2010.

\leavevmode\hypertarget{ref-halsey2015}{}%
{[}16{]} L. G. Halsey, D. Curran-Everett, S. L. Vowler, and G. B.
Drummond, ``The fickle p value generates irreproducible results,''
\emph{Nature methods}, vol. 12, no. 3, pp. 179--185, 2015.

\leavevmode\hypertarget{ref-vishwanath2009}{}%
{[}17{]} K. Vishwanath, H. Yuan, W. T. Barry, M. W. Dewhirst, and N.
Ramanujam, ``Using optical spectroscopy to longitudinally monitor
physiological changes within solid tumors,'' \emph{Neoplasia}, vol. 11,
no. 9, pp. 889--900, 2009.

\leavevmode\hypertarget{ref-pinheiro2006}{}%
{[}18{]} J. Pinheiro and D. Bates, \emph{Mixed-effects models in s and
s-plus}. Springer Science \& Business Media, 2006.

\leavevmode\hypertarget{ref-rose2012}{}%
{[}19{]} N. L. Rose, H. Yang, S. D. Turner, and G. L. Simpson, ``An
assessment of the mechanisms for the transfer of lead and mercury from
atmospherically contaminated organic soils to lake sediments with
particular reference to scotland, uk,'' \emph{Geochimica et Cosmochimica
Acta}, vol. 82, pp. 113--135, 2012.

\leavevmode\hypertarget{ref-pedersen2019}{}%
{[}20{]} E. J. Pedersen, D. L. Miller, G. L. Simpson, and N. Ross,
``Hierarchical generalized additive models in ecology: An introduction
with mgcv,'' \emph{PeerJ}, vol. 7, p. e6876, 2019.

\leavevmode\hypertarget{ref-simpson2018}{}%
{[}21{]} G. L. Simpson, ``Modelling palaeoecological time series using
generalised additive models,'' \emph{Frontiers in Ecology and
Evolution}, vol. 6, p. 149, 2018.

\leavevmode\hypertarget{ref-yang2012}{}%
{[}22{]} L. Yang, G. Qin, N. Zhao, C. Wang, and G. Song, ``Using a
generalized additive model with autoregressive terms to study the
effects of daily temperature on mortality,'' \emph{BMC Medical Research
Methodology}, vol. 12, no. 1, p. 165, 2012.

\leavevmode\hypertarget{ref-beck1998}{}%
{[}23{]} N. Beck and S. Jackman, ``Beyond linearity by default:
Generalized additive models,'' \emph{American Journal of Political
Science}, pp. 596--627, 1998.

\leavevmode\hypertarget{ref-wood2017}{}%
{[}24{]} S. N. Wood, \emph{Generalized additive models: An introduction
with r}. CRC press, 2017.

\leavevmode\hypertarget{ref-wood2011}{}%
{[}25{]} S. N. Wood, N., Pya, and B. S"afken, ``Smoothing parameter and
model selection for general smooth models (with discussion),''
\emph{Journal of the American Statistical Association}, vol. 111, pp.
1548--1575, 2016.

\leavevmode\hypertarget{ref-wood2016}{}%
{[}26{]} S. N. Wood, N., Pya, and B. S"afken, ``Smoothing parameter and
model selection for general smooth models (with discussion),''
\emph{Journal of the American Statistical Association}, vol. 111, pp.
1548--1575, 2016.

\leavevmode\hypertarget{ref-haverkamp2017}{}%
{[}27{]} N. Haverkamp and A. Beauducel, ``Violation of the sphericity
assumption and its effect on type-i error rates in repeated measures
anova and multi-level linear models (mlm),'' \emph{Frontiers in
psychology}, vol. 8, p. 1841, 2017.

\leavevmode\hypertarget{ref-nelder1972}{}%
{[}28{]} J. A. Nelder and R. W. Wedderburn, ``Generalized linear
models,'' \emph{Journal of the Royal Statistical Society: Series A
(General)}, vol. 135, no. 3, pp. 370--384, 1972.

\leavevmode\hypertarget{ref-wolfinger1996}{}%
{[}29{]} R. D. Wolfinger, ``Heterogeneous variance: Covariance
structures for repeated measures,'' \emph{Journal of agricultural,
biological, and environmental statistics}, pp. 205--230, 1996.

\leavevmode\hypertarget{ref-west2014}{}%
{[}30{]} B. T. West, K. B. Welch, and A. T. Galecki, \emph{Linear mixed
models: A practical guide using statistical software}. CRC Press, 2014.

\leavevmode\hypertarget{ref-weiss2005}{}%
{[}31{]} R. E. Weiss, \emph{Modeling longitudinal data}. Springer
Science \& Business Media, 2005.

\leavevmode\hypertarget{ref-geisser1958}{}%
{[}32{]} S. Geisser, S. W. Greenhouse, and others, ``An extension of
box's results on the use of the \(F\) distribution in multivariate
analysis,'' \emph{The Annals of Mathematical Statistics}, vol. 29, no.
3, pp. 885--891, 1958.

\leavevmode\hypertarget{ref-huynh1976}{}%
{[}33{]} H. Huynh and L. S. Feldt, ``Estimation of the box correction
for degrees of freedom from sample data in randomized block and
split-plot designs,'' \emph{Journal of educational statistics}, vol. 1,
no. 1, pp. 69--82, 1976.

\leavevmode\hypertarget{ref-maxwell2017}{}%
{[}34{]} S. E. Maxwell, H. D. Delaney, and K. Kelley, \emph{Designing
experiments and analyzing data: A model comparison perspective}.
Routledge, 2017.

\leavevmode\hypertarget{ref-keselman2001}{}%
{[}35{]} H. Keselman, J. Algina, and R. K. Kowalchuk, ``The analysis of
repeated measures designs: A review,'' \emph{British Journal of
Mathematical and Statistical Psychology}, vol. 54, no. 1, pp. 1--20,
2001.

\leavevmode\hypertarget{ref-scheffer2002}{}%
{[}36{]} J. Scheffer, ``Dealing with missing data,'' 2002.

\leavevmode\hypertarget{ref-potthoff2006}{}%
{[}37{]} R. F. Potthoff, G. E. Tudor, K. S. Pieper, and V. Hasselblad,
``Can one assess whether missing data are missing at random in medical
studies?'' \emph{Statistical methods in medical research}, vol. 15, no.
3, pp. 213--234, 2006.

\leavevmode\hypertarget{ref-ma2012}{}%
{[}38{]} Y. Ma, M. Mazumdar, and S. G. Memtsoudis, ``Beyond
repeated-measures analysis of variance: Advanced statistical methods for
the analysis of longitudinal data in anesthesia research,''
\emph{Regional Anesthesia \& Pain Medicine}, vol. 37, no. 1, pp.
99--105, 2012.

\leavevmode\hypertarget{ref-hastie1987}{}%
{[}39{]} T. Hastie and R. Tibshirani, ``Generalized additive models:
Some applications,'' \emph{Journal of the American Statistical
Association}, vol. 82, no. 398, pp. 371--386, 1987.

\leavevmode\hypertarget{ref-wegman1983}{}%
{[}40{]} E. J. Wegman and I. W. Wright, ``Splines in statistics,''
\emph{Journal of the American Statistical Association}, vol. 78, no.
382, pp. 351--365, 1983.

\end{document}
